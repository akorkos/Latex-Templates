\documentclass[oneside]{book}
\usepackage{preamble}

\begin{document}
    \begin{titlepage}

    \begin{figure}[!htb]
        \centering
        \begin{minipage}{0.45\textwidth}
            \centering
            \includegraphics[width=0.4\textwidth]{images/logo_csd.png} % first figure itself
        \end{minipage}\hfill
        \begin{minipage}{0.45\textwidth}
            \centering
            \includegraphics[width=0.5\textwidth]{images/logo_auth.png} % second figure itself
        \end{minipage}
    \end{figure}
     
    \begin{center}
        \LARGE{Αριστοτέλειο Πανεπιστήμιο Θεσσαλονίκης}
        \vspace{5mm}
        \\ \Large{Τμήμα Πληροφορικής}
    \end{center}
    
    \vspace*{\fill}
    
    \begin{center}
        \Rule \\[0.4cm]
        { \LARGE 
            \textbf{Τεχνική αναφορά για NCO-03-02}\\[0.4cm]
            \emph{2η υποχρεωτική εργασία}\\[0.4cm]
        }
        \Rule \\[0.4cm]
    \end{center}
    
    \vspace*{\fill}
    
    \begin{center}
        Αλέξανδρος Κόρκος \\
        \textbf{\href{mailto:alexkork@csd.auth.gr}{alexkork@csd.auth.gr}}\\
        \textbf{3870}
        \Rule \\[0.4cm]
        Θεσσαλονίκη, \today
    \end{center}
    
\end{titlepage}

    \begin{frame}{}
    \begin{block}{Άδεια}
        \begin{center}
            \href{https://creativecommons.org/licenses/by-nc-sa/4.0/deed.el}{\includegraphics[scale=0.2]{images/cc.png}} \\
            Το έργο αυτό διατίθεται υπό τους όρους της άδειας \textbf{\href{https://creativecommons.org/licenses/by-nc-sa/4.0/deed.el}{Create Commons "Αναφορά Δημιουργού - Μη Εμπορική Χρήση - Παρόμοια Διανομή 4.0 Διεθνές"}}. \\ 
        \end{center}
    \end{block}
\end{frame}
    
    \tableofcontents
    
    \chapter{Τα όνειρα κοστίζουν}
    
    \lipsum[1]

    \begin{figure}[H]
        \includegraphics[scale=0.2]{images/skg.jpg}
        \caption{Θεσσαλονίκη}
        \label{fig:skg}
    \end{figure}

    \section{Βλέπεις λάθος εικόνα}

    \lipsum[1-3]

    Εικόνα \ref{fig:skg}.

    \chapter{Where the trees meet the freeway}

    \begin{table}[H]
        \begin{tabular}{|c|c|c|c|c|c|c|c|c|c|c|c|c|c|c|c|c|} 
			\hline
			$A$ & $B$ & $C$ & $D$ & $E$ & $F$ & $G$ & $H$ & $K$ & $L$ & $M$ & $N$ & $O$ & $P$ & $Q$ & $R$ & $S$ \\
			\hline
			1 & 2 & 3 & 4 & 5 & 6 & 7 & 8 & 9 & 10 & 11 & 14 & 15 & 16 & 17 & 18 & 19 \\
			\hline
		\end{tabular}
        \caption{Πίνακας}
    \end{table}    

    \lipsum[1-2]

    \section{9mm}

    \lipsum[1-2]

    \begin{listing}[H]
    \begin{minted}{python}
    def ypologismos(arithmosTemaxiwn):
        if arithmosTemaxiwn <= 3:
            return arithmosTemaxiwn * 120
        elif arithmosTemaxiwn <= 6:
            return 3 * 120 + (arithmosTemaxiwn - 3) * 100
        else:
            return 3 * 120 + 3 * 100 + (arithmosTemaxiwn - 6) * 70

    esodaKatasthmatos = 0
    plithosPelatwnOver10 = 0
    for costumer in range(50):
        arithmosTemaxiwn = int(input("Αριθμός των τεμαχίων που αγοράσατε: "))
        if arithmosTemaxiwn > 10:
            plithosPelatwnOver10 += 1
        xrewshPelath = ypologismos(arithmosTemaxiwn)
        esodaKatasthmatos += xrewshPelath
        print "Χρέωση του πελάτη: ", xrewshPelath

    print "Συνολικά έσοδα του καταστήματος: ", esodaKatasthmatos
    print "Ποσοστό των πελατών που αγόρασαν πάνω από 10 τεμάχια: ", \ 
    (plithosPelatwnOver10 / 50.0) * 100.0
    \end{minted}
    \caption{θέμα Γ}
    \end{listing}
        
    \section{First class}

    \begin{equation}\label{asterisq2}
        \frac{q^{1+2f(n)}}{1.5}<q - \frac{1}{2}q^{n/(n+1)+f_q(n)}.
    \end{equation}
        
    With $L$ we denote the full rank lattice of rank $n+1$ generated by the vectors
    ${\bf b}_0 = (-1,A_1,...,A_{n}),$
    ${\bf b}_j = (0,0,...,q,...,0),$ where $q$ is in the position $j+1$ for $j=1,...,n.$ Then, for all non-zero ${\bf v}\in L$ we have
    $$||{\bf v}||>\frac{q^{n/(n+1)+f_q(n)}}{2}.$$

    \section{2 Seater}

    \begin{center}
        \scalebox{20}{\faIcon{linux}}
    \end{center}

    \begin{example}
        \lipsum[1]
    \end{example}

    \begin{thebibliography}{3} 
    \bibitem{sauer} Timothy Sauer, \emph{Numerical Analysis 2nd ed.}, Pearson Education, 2012.
    \bibitem{aran} Γ. Σ. Παπαγεωργίου, Χ. Γρ. Τσίτσουρας, \emph{Αριθμητική Ανάλυση με εφαρμογές σε Mathematica και Matlab}, Εκδόσεις Τσότρας, 2015.
    \bibitem{jk} Jaan Kiusalaas, \emph{NUMERICAL METHODS IN ENGINEERING WITH Python},
    Cambridge University Press, 2005.
\end{thebibliography}

\end{document}