\documentclass[aspectratio=169]{beamer}
\usepackage{preamble}

\title{Τεχνική αναφορά για NCO‐03‐02}
\subtitle{2η υποχρεωτική εργασία}
\author{Αλέξανδρος Κόρκος}
\institute{Αριστοτέλειο Πανεπιστήμιο Θεσσαλονίκης \\ Τμήμα Πληροφορικής}
\date{\today}

\begin{document}

    \frame{\titlepage}

    \begin{frame}{}
    \begin{block}{Άδεια}
        \begin{center}
            \href{https://creativecommons.org/licenses/by-nc-sa/4.0/deed.el}{\includegraphics[scale=0.2]{images/cc.png}} \\
            Το έργο αυτό διατίθεται υπό τους όρους της άδειας \textbf{\href{https://creativecommons.org/licenses/by-nc-sa/4.0/deed.el}{Create Commons "Αναφορά Δημιουργού - Μη Εμπορική Χρήση - Παρόμοια Διανομή 4.0 Διεθνές"}}. \\ 
        \end{center}
    \end{block}
\end{frame}

    \begin{frame}{Περιεχόμενα}
        \tableofcontents
    \end{frame}

    \section{Τα όνειρα κοστίζουν}

    \begin{frame}{Social Psychology}
        \lipsum[1]
    \end{frame}

    \begin{frame}
        \begin{figure}
            \includegraphics[width=0.25\textwidth]{images/skg.jpg}
            \caption{Θεσσαλονίκη}
            \label{fig:skg}
        \end{figure}
    \end{frame}

    \begin{frame}{Bullet Points}
        \begin{itemize}
            \item Lorem ipsum dolor sit amet, consectetur adipiscing elit
            \item Aliquam blandit faucibus nisi, sit amet dapibus enim tempus eu
            \item Nulla commodo, erat quis gravida posuere, elit lacus lobortis est, quis porttitor odio mauris at libero
            \item Nam cursus est eget velit posuere pellentesque
            \item Vestibulum faucibus velit a augue condimentum quis convallis nulla gravida
        \end{itemize}
    \end{frame}

    \section{Advanced Criminal Law}

    \begin{frame}[fragile]{Code}
        \begin{listing}[H]
            \begin{minted}{python}
            def ypologismos(arithmosTemaxiwn):
                if arithmosTemaxiwn <= 3:
                    return arithmosTemaxiwn * 120
                elif arithmosTemaxiwn <= 6:
                    return 3 * 120 + (arithmosTemaxiwn - 3) * 100
                else:
                    return 3 * 120 + 3 * 100 + (arithmosTemaxiwn - 6) * 70
            \end{minted}
            \caption{θέμα Γ}
        \end{listing}
    \end{frame}

    \begin{frame}{Table}
        \begin{table}
            \begin{tabular}{|l|l|l|}
                \hline
                \textbf{Treatments} & \textbf{Response 1} & \textbf{Response 2} \\ \hline
                Treatment 1         & 0.0003262           & 0.562               \\
                Treatment 2         & 0.0015681           & 0.910               \\
                Treatment 3         & 0.0009271           & 0.296               \\ \hline
            \end{tabular}
            \caption{2310}
        \end{table}
    \end{frame}

    \section{Physical Education}

    \begin{frame}{Maths}
        \begin{equation}\label{asterisq2}
            \frac{q^{1+2f(n)}}{1.5}<q - \frac{1}{2}q^{n/(n+1)+f_q(n)}.
        \end{equation}
            
        With $L$ we denote the full rank lattice of rank $n+1$ generated by the vectors
        ${\bf b}_0 = (-1,A_1,...,A_{n}),$
        ${\bf b}_j = (0,0,...,q,...,0),$ where $q$ is in the position $j+1$ for $j=1,...,n.$ Then, for all non-zero ${\bf v}\in L$ we have
        $$||{\bf v}||>\frac{q^{n/(n+1)+f_q(n)}}{2}.$$
    \end{frame}

    \begin{thebibliography}{3} 
    \bibitem{sauer} Timothy Sauer, \emph{Numerical Analysis 2nd ed.}, Pearson Education, 2012.
    \bibitem{aran} Γ. Σ. Παπαγεωργίου, Χ. Γρ. Τσίτσουρας, \emph{Αριθμητική Ανάλυση με εφαρμογές σε Mathematica και Matlab}, Εκδόσεις Τσότρας, 2015.
    \bibitem{jk} Jaan Kiusalaas, \emph{NUMERICAL METHODS IN ENGINEERING WITH Python},
    Cambridge University Press, 2005.
\end{thebibliography}

    \begin{frame}
        \Huge{\centerline{Ερωτήσεις;}} 
        \normalsize{\centerline{\href{mailto:alexkork@csd.auth.gr}{alexkork@csd.auth.gr}}}
    \end{frame}
    
\end{document}